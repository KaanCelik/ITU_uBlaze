\chapter{Introduction}
\section{Introduction}
As mobile technologies advance further connectivity becomes increasingly more important. Latest developments in smart phone industry created an environment where many devices communicate with each other. In this context Bluetooth technology makes it possible to connect together various devices with diverse sizes and complexities. On the other hand, FPGAs power many consumer electronics and industrial applications. \cite{eetimes-fpga} Internet of things has become a growing industry on its own with many practical applications and even FPGAs solely targeting this market are being developed. \cite{eetimes-lattice} Combining the technologies described above gives way to innovative applications and this is the main motivation behind this study.

In light of these developments this thesis explores a proof of concept application which establishes communication between a smart phone and a FPGA based system.

\section{Objective}
Primary objective of this work is to develop an application where an Android App can communicate with a FPGA based system using Bluetooth connection and manipulate data on the FPGA.

\section{Scope}
Requirements for this project are as follows:
\begin{itemize}
	\item Android App should be able to transfer 16 8-bits numbers to FPGA over Bluetooth
	\item Android App should have an user friendly interface
	\item FPGA based system can receive data sent from Android App
	\item FPGA based system can save the data to a memory block which can be accessed by other FPGA modules
	\item Data on memory block can be observed from a display unit
\end{itemize}
